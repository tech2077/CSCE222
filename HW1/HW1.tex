\documentclass{article}
\usepackage{algorithm,amsmath,amsthm,latexsym,paralist,url}
\usepackage[margin=1in]{geometry}
\usepackage[noend]{algpseudocode}

\theoremstyle{definition}
\newtheorem{problem}{Problem}
\newtheorem{solution}{Solution}
\newtheorem*{resources}{Resources}


\newcommand{\honor}{\noindent \textbf{Aggie Honor Statement: }On my honor as an Aggie, I have neither
  given nor received any unauthorized aid on any portion of the academic work included in this assignment.
}

 
\newcommand{\checklist}{\noindent\textbf{Checklist:}
\begin{compactenum}
\item Did you abide by the Aggie Honor Code?
\item Did you solve all problems and start a new page for each? 
\item Did you submit the PDF to eCampus?
\end{compactenum}
}

\newcommand{\problemset}[1]{\begin{center}\textbf{Problem Set #1}\end{center}}
\newcommand{\duedate}[1]{\begin{quote}\textbf{Due: #1} on eCampus (\url{ecampus.tamu.edu}).\end{quote}}
\newcommand{\mysectionnumber}[0]{503}

\title{CSCE 222: Discrete Structures for Computing\\Section \mysectionnumber\\Fall 2016}
\author{Matthew Skolaut}

\begin{document}

\maketitle

\problemset{1}

\duedate{4 September 2016 (Sunday) before 11:59 p.m.}

\bigskip

% Algorithms
\begin{problem} (20 points)\\
You have $n$ coins, exactly one of which is counterfeit.  You know counterfeit coins weigh more than authentic coins.  Devise an algorithm for finding the counterfeit coin using a balance scale\footnote{A balance scale compares the weight of objects placed on it.  The result of the comparison is either left side heavier, right side heavier, or both sides equal.}.  Express your algorithm in pseudocode.  For $n=12$, how many weighings does your algorithm use? 
\end{problem}

\begin{solution}
Algorithm
\begin{algorithm}
\caption{Finding counterfeit coin}
\begin{algorithmic}[1]
\Procedure{FindCounterfeit}{$c_0,\ c_1,\,.\,.\,.\,,\ c_n:$ list of coins}
	\For {$i := 0$ to $n-1$}
		\State weigh $c_i$ on left and $c_{i+1}$ on right on scale
		\If {scale tips left}
			\State \Return $c_i$
		\ElsIf {scale tips right}
			\State \Return $c_{i+1}$
		\EndIf
	\EndFor
\EndProcedure
\end{algorithmic}
\end{algorithm}
\end{solution}

\newpage

% Algorithms
\begin{problem} (20 points)\\
Devise an algorithm that takes as input a list of $n$ integers in unsorted order, where the integers are not necessarily distinct, and outputs the location (index of first element) and length of the longest contiguous non-decreasing subsequence in the list.  If there is a tie, it outputs the location of the first such subsequence.  Express your algorithm in pseudocode.  For the list $9,7,9,4,5,8,1,0,5,9$, what is the algorithm's output? How many comparison operations between elements of the list are used?
\end{problem}

\begin{solution}
Algorithm
\begin{algorithm}
\caption{Finding counterfeit coin}
\begin{algorithmic}[1]
\Procedure{FindCounterfeit}{$c_0,\ c_1,\,.\,.\,.\,,\ c_n:$ list of coins}
	\For {$i := 0$ to $n-1$}
		\State weigh $c_i$ on left and $c_{i+1}$ on right on scale
		\If {scale tips left}
			\State \Return $c_i$
		\ElsIf {scale tips right}
			\State \Return $c_{i+1}$
		\EndIf
	\EndFor
\EndProcedure
\end{algorithmic}
\end{algorithm}
\end{solution}

\newpage

% Growth of Functions - Big O
\begin{problem} (20 points)\\
Arrange the following functions in order such that each function is big-$O$ of the next function:  $2\cdot3^n$, $3n!$, $2019\log{n}$, $\displaystyle \frac{n^3}{10^6}$, $n\log{n}$,  $\sqrt{n}$, $3\cdot2^n$.  Prove your answer is correct by giving the witnesses for each pair of consecutive functions.
\end{problem}

\begin{solution}
\end{solution}

\newpage

% Growth of Functions - Big O
\begin{problem} (20 points)\\
For each of the following functions, give a big-$O$ estimate, including witnesses, using a simple function $g(n)$ of the smallest order:
\begin{enumerate}
\item $(n^2+8)(n+1)$
\item $(n\log{n} + 1)^2+(\log{n}+1)(n^2+1)$
\item $\displaystyle n^{2^n}+n^{n^2}$
\item $\displaystyle \frac{n^4+5\log{n}}{n^3+1}$
\item $2n^4+7n^3+5n+3$
\end{enumerate}
\end{problem}

\begin{solution}
\end{solution}


\bigskip
\honor

\bigskip
\checklist
\end{document}
