\documentclass{article}
\usepackage{algorithm,amsmath,ntheorem,latexsym,paralist,url}
\usepackage[margin=1in]{geometry}
\usepackage[noend]{algpseudocode}

\theoremstyle{plain} 
\theorembodyfont{\normalfont} 
\newtheorem{problem}{Problem}
\theoremstyle{break} 
\theorembodyfont{\normalfont} 
\newtheorem{solution}{Solution}
\newtheorem*{resources}{Resources}


\newcommand{\honor}{\noindent \textbf{Aggie Honor Statement: }On my honor as an Aggie, I have neither
  given nor received any unauthorized aid on any portion of the academic work included in this assignment.
}

 
\newcommand{\checklist}{\noindent\textbf{Checklist:}
\begin{compactenum}
\item Did you abide by the Aggie Honor Code?
\item Did you solve all problems and start a new page for each? 
\item Did you submit the PDF to eCampus?
\end{compactenum}
}

\newcommand{\problemset}[1]{\begin{center}\textbf{Problem Set #1}\end{center}}
\newcommand{\duedate}[1]{\begin{quote}\textbf{Due: #1} on eCampus (\url{ecampus.tamu.edu}).\end{quote}}
\newcommand{\mysectionnumber}[0]{503}

\title{CSCE 222: Discrete Structures for Computing\\Section \mysectionnumber\\Fall 2016}
\author{Matthew Skolaut}

\begin{document}

\maketitle

\problemset{2}

\duedate{11 September 2016 (Sunday) before 11:59 p.m.}

\bigskip

% Growth of Functions - Big Theta
\begin{problem} (20 points)\\
For each of the following functions, determine whether that function is of the same order as $n^2$ either by finding witnesses or showing that sufficient witnesses do not exist:
\begin{enumerate}
\item $13n+12$
\item $n^2+1000 n\log{n}$
\item $3^n$
\item $3n^2+n-5$
\item $\displaystyle \frac{n^3+2n^2-n+3}{4n}$
\end{enumerate}
\end{problem}

\begin{solution}
\end{solution}

\newpage

% Complexity of Algorithms
\begin{problem} (20 points)\\
Do Supplementary Exercise 29 of Chapter 3 (page 234).
\end{problem}

\begin{solution}
\end{solution}

\newpage

% Propositional Logic
\begin{problem} (20 points)\\
Do  Exercise 31 of Chapter 1.1 (page 15).
\end{problem}

\begin{solution}
\end{solution}

\newpage

% Applications of Propositional Logic
\begin{problem} (20 points)\\
Do  Exercises 19, 21, and 23 of Chapter 1.2 (page 23).
\end{problem}

\begin{solution}
\end{solution}

\newpage

% Propositional Equivalences
\begin{problem} (20 points)\\
Do  Exercises 50 and 51 of Chapter 1.3 (page 36).
\end{problem}

\begin{solution}
\end{solution}

\bigskip
\honor

\bigskip
\checklist
\end{document}
